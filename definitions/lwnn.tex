\documentclass[10pt]{article}
\usepackage[cm]{fullpage}
\usepackage{pdflscape}
\usepackage{longtable}
\usepackage{bcprules}
\usepackage{amsmath}
\usepackage{amsfonts}
\usepackage{amssymb}
\usepackage{amsthm}
\usepackage{color}
\usepackage{verbatim}
\usepackage{xspace}
\usepackage{xargs}
\usepackage{multirow}
\usepackage{graphicx}
\usepackage{caption}
\usepackage{booktabs}
\usepackage{stmaryrd}
\usepackage[figuresleft]{rotating}
\usepackage{galois}
\usepackage{array}
\usepackage{mdframed}
\usepackage{bm}
\usepackage{txfonts}

%% math style columns for tables
\newcolumntype{R}{>{$}r<{$}}
\newcolumntype{L}{>{$}l<{$}}
\newcolumntype{C}{>{$}c<{$}}

%% misc
\newcommand{\alt}{\ \mid\ }
\newcommand{\lalt}{\ \ \ \alt}
\newcommand{\anywhere}[1]{\ensuremath{\mbox{#1}}}
\newcommand{\kw}[1]{\anywhere{\sffamily{\bfseries\small {#1}}}}
\newcommand{\ignore}[1]{}
\newcommand{\vsp}{\vspace{.1in}}
\newcommand{\nvsp}{\vspace{-.1in}}
\newcommand{\mtt}[1]{\ensuremath{\mathit{#1}}}
\newcommand{\mrm}[1]{\ensuremath{\mathrm{#1}}}
\newcommand{\ttt}[1]{\ensuremath{\mbox{\small \texttt{#1}}}}
\newcommand{\red}[1]{{\color{red}{#1}}}
\newcommand{\blue}[1]{{\color{blue}{#1}}}
\newcommand{\green}[1]{{\color{green}{#1}}}
\newcommand{\note}[1]{\red{\textbf{[#1]}}}
\newcommand{\something}{\ensuremath{\,\cdot\,}}

%% math
\newcommand{\seq}[2][]{\ensuremath{\vec{#2}}\xspace}
\newcommand{\wseq}[2][]{\ensuremath{\overrightarrow{#2}}\xspace}
\newcommand{\uset}[1]{\ensuremath{\overline{#1}}\xspace}
\newcommand{\power}{\mathcal{P}\xspace}
\newcommand{\proj}[2]{\ensuremath{\pi_{#1}(#2)}\xspace}
\newcommand{\nada}{\ensuremath{\bullet}\xspace}
\newcommand{\sopt}[1]{#1^?\xspace}
\newcommand{\sseq}[1]{#1^\star\xspace}
\newcommand{\sseqp}[1]{#1^+\xspace}
\newcommand{\fold}{\ttt{foldl}\xspace}
\newcommand{\lte}{\ensuremath{\sqsubseteq}\xspace}
\newcommand{\gte}{\ensuremath{\sqsupseteq}}

%% types
\newcommand{\intt}{\ttt{int}\xspace}
\newcommand{\boolt}{\ttt{bool}\xspace}
\newcommand{\strt}{\ttt{string}\xspace}
\newcommand{\nullt}{\ttt{null}\xspace}
\newcommand{\TopClass}{\ttt{TopClass}\xspace}

%% syntactic domains
\newcommand{\Int}{\ensuremath{\mathbb{Z}}\xspace}
\newcommand{\Nat}{\ensuremath{\mathbb{N}}\xspace}
\newcommand{\Bool}{\mtt{Bool}\xspace}
\newcommand{\String}{\mtt{String}\xspace}
\newcommand{\Variable}{\mtt{Variable}\xspace}
\newcommand{\BinOp}{\mtt{BinaryOp}\xspace}
\newcommand{\Exp}{\mtt{Exp}\xspace}
\newcommand{\Stmt}{\mtt{Stmt}\xspace}
\newcommand{\Program}{\mtt{Program}\xspace}
\newcommand{\Method}{\mtt{Method}\xspace}
\newcommand{\Class}{\mtt{Class}\xspace}
\newcommand{\Type}{\mtt{Type}\xspace}
\newcommand{\CName}{\mtt{ClassName}\xspace}
\newcommand{\MName}{\mtt{MethodName}\xspace}

%% syntactic metavariables and terms
\newcommand{\str}{\mtt{str}\xspace}
\newcommand{\class}{\mtt{class}\xspace}
\newcommand{\true}{\kw{true}\xspace}
\newcommand{\false}{\kw{false}\xspace}
\newcommand{\self}{\ttt{self}\xspace}
\newcommand{\stmt}{\mtt{s}\xspace}
\newcommand{\typ}{\ensuremath{\tau}}
\newcommand{\cname}{\mtt{cn}\xspace}
\newcommand{\mname}{\mtt{mn}\xspace}
\newcommand{\binop}{\ensuremath{\oplus}\xspace}
\newcommand{\cond}[3]{\ensuremath{\kw{if }#1\; #2 \kw{ else } #3}\xspace}
\newcommand{\while}[2]{\ensuremath{\kw{while }#1\; #2}\xspace}
\newcommand{\nullv}{\kw{null}\xspace}

%% concrete semantic domain
\newcommand{\Heap}{\mtt{Heap}\xspace}
\newcommand{\Local}{\mtt{Locals}\xspace}
\newcommand{\Value}{\mtt{Value}\xspace}
\newcommand{\Address}{\mtt{Address}\xspace}
\newcommand{\State}{\mtt{State}\xspace}
\newcommand{\Kont}{\mtt{Kont}\xspace}
\newcommand{\Classes}{\mtt{ClassDefs}\xspace}
\newcommand{\Object}{\mtt{Object}\xspace}
\newcommand{\Ref}{\ensuremath{\mtt{Reference}}\xspace}

%% concrete semantic objects
\newcommand{\state}{\ensuremath{\varsigma}\xspace}
\newcommand{\local}{\ensuremath{\rho}\xspace}
\newcommand{\heap}{\ensuremath{\sigma}\xspace}
\newcommand{\dom}{\ensuremath{\ttt{dom}}\xspace}
\newcommand{\alloc}{\ensuremath{\ttt{alloc}}\xspace}
\newcommand{\eeval}{\ensuremath{\eta}\xspace}
\newcommand{\evalTo}[1]{\ensuremath{\llbracket #1 \rrbracket}\xspace}
\newcommand{\step}{\ensuremath{\mathcal{F}}\xspace}
\newcommand{\classes}{\ensuremath{\theta}\xspace}
\newcommand{\kont}{\kappa\xspace}
\newcommand{\sstmtk}[1]{\kw{stmtK}\;#1\xspace}
\newcommand{\swhilek}[2]{\kw{whileK}\;#1 \times #2\xspace}
\newcommand{\sretk}[3]{\kw{retK}\;#1 \times #2 \times #3\xspace}
\newcommand{\stmtk}[1]{\kw{stmtK}(#1)\xspace}
\newcommand{\whilek}[2]{\kw{whileK}(#1,\, #2)\xspace}
\newcommand{\retk}[3]{\kw{retK}(#1,\, #2,\, #3)\xspace}

%% concrete helper functions
\newcommand{\tostmtk}{\ttt{toSK}\xspace}
\newcommand{\funcall}{\ttt{call}\xspace}
\newcommand{\construct}{\ttt{construct}\xspace}
\newcommand{\initstate}{\ttt{initstate}\xspace}
\newcommand{\initclass}{\ttt{initclass}\xspace}
\newcommand{\defval}{\ttt{defaultvalue}\xspace}
\newcommand{\head}{\ttt{head}\xspace}

%% abstract semantic domain
\newcommand{\AInt}{\ensuremath{\Int^\sharp}\xspace}
\newcommand{\ABool}{\ensuremath{\Bool^\sharp}\xspace}
\newcommand{\AString}{\ensuremath{\String^\sharp}\xspace}
\newcommand{\ABinOp}{\ensuremath{\mtt{BinaryOp}^\sharp}\xspace}
\newcommand{\AHeap}{\ensuremath{\mtt{Heap}^\sharp}\xspace}
\newcommand{\ALocal}{\ensuremath{\mtt{Locals}^\sharp}\xspace}
\newcommand{\AValue}{\ensuremath{\mtt{Value}^\sharp}\xspace}
\newcommand{\AAddress}{\ensuremath{\mtt{Address}^\sharp}\xspace}
\newcommand{\AState}{\ensuremath{\mtt{State}^\sharp}\xspace}
\newcommand{\AKont}{\ensuremath{\mtt{Kont}^\sharp}\xspace}
\newcommand{\AClasses}{\ensuremath{\mtt{ClassDefs}^\sharp}\xspace}
\newcommand{\AObject}{\ensuremath{\mtt{Object}^\sharp}\xspace}
\newcommand{\ARef}{\ensuremath{\mtt{Reference}^\sharp}\xspace}

%% abstract semantic objects
\newcommand{\astate}{\ensuremath{\hat{\state}}\xspace}
\newcommand{\alocal}{\ensuremath{\hat{\local}}\xspace}
\newcommand{\aheap}{\ensuremath{\hat{\heap}}\xspace}
\newcommand{\aalloc}{\ensuremath{\widehat{\alloc}}\xspace}
\newcommand{\aeeval}{\ensuremath{\eeval}^\sharp\xspace}
\newcommand{\aevalTo}[1]{\ensuremath{\llbracket #1 \rrbracket^\sharp}\xspace}
\newcommand{\astep}{\ensuremath{\hat{\step}}\xspace}
\newcommand{\aclasses}{\ensuremath{\hat{\classes}}\xspace}
\newcommand{\akont}{\ensuremath{\hat{\kont}}\xspace}
\newcommand{\asstmtk}[1]{\widehat{\kw{stmtK}}\;#1\xspace}
\newcommand{\aswhilek}[2]{\widehat{\kw{whileK}}\;#1 \times #2\xspace}
\newcommand{\asretk}[3]{\widehat{\kw{retK}}\;#1 \times #2 \times #3\xspace}
\newcommand{\asfink}{\widehat{\kw{finK}}\xspace}
\newcommand{\astmtk}[1]{\widehat{\kw{stmtK}}(#1)\xspace}
\newcommand{\awhilek}[2]{\widehat{\kw{whileK}}(#1,\, #2)\xspace}
\newcommand{\aretk}[3]{\widehat{\kw{retK}}(#1,\, #2,\, #3)\xspace}
\newcommand{\afink}{\widehat{\kw{finK}}\xspace}
\newcommand{\av}{\ensuremath{\hat{v}}\xspace}
\newcommand{\ao}{\ensuremath{\hat{o}}\xspace}
\newcommand{\aad}{\ensuremath{\hat{a}}\xspace}
\newcommand{\anullv}{\ensuremath{\widehat{\kw{null}}}\xspace}
\newcommand{\an}{\ensuremath{\hat{n}}\xspace}
\newcommand{\ab}{\ensuremath{\hat{b}}\xspace}
\newcommand{\astr}{\ensuremath{\widehat{\mtt{str}}}\xspace}
\newcommand{\abinop}{\ensuremath{\hat{\oplus}}\xspace}
\newcommand{\aref}{\ensuremath{\hat{r}}}

%% abstract helper functions
\newcommand{\atostmtk}{\ensuremath{\ttt{toSK}^\sharp}\xspace}
\newcommand{\afuncall}{\ensuremath{\ttt{call}^\sharp}\xspace}
\newcommand{\aconstruct}{\ensuremath{\ttt{construct}^\sharp}\xspace}
\newcommand{\ainitstate}{\ensuremath{\ttt{initstate}^\sharp}\xspace}
\newcommand{\ainitclass}{\ensuremath{\ttt{initclass}^\sharp}\xspace}
\newcommand{\adefval}{\ensuremath{\ttt{defaultvalue}^\sharp}\xspace}
\newcommand{\aupdate}{\ensuremath{\ttt{update}^\sharp}\xspace}
\newcommand{\alookup}{\ensuremath{\ttt{lookup}^\sharp}\xspace}

%%%%%%%%%%%%%%%%%%%%%%%%%%%%%%%%%%%%%%%%%%%%%%%%%%%%%%%%%%%%%%%%%%%%%%%%%%%%%%%%

\ignore{
  \newcommand{\aeeval}{\ensuremath{\eta^\sharp}\xspace}
  \newcommand{\dc}{\ensuremath{\underline{\;\;}}\xspace}
  \newcommand{\yield}[1]{\ensuremath{\!\!\downharpoonright_{#1}}}
  \newcommand{\pto}{\mat{\rightharpoonup}}
  \newcommand{\bigjoin}{\mat{\bigsqcup}}
  \newcommand{\lte}{\mat{\sqsubseteq}\xspace}
  \newcommand{\gte}{\mat{\sqsupseteq}}
  \newcommand{\lattice}{\mat{\mathcal{L}}\xspace}
  \newcommand{\alattice}{\mat{\lattice^\sharp}\xspace}
  \newcommand{\widen}{\mat{\triangledown}\xspace}
  \newcommand{\pre}{\mat{\preceq}}
  \newcommand{\sube}{\mat{\subseteq}}
}


\begin{document}

\title{\textbf{\textsf{Lwnn}} Concrete and Abstract Semantics}
\author{CS 260, Fall 2013}
\date{}

\maketitle

\section{Lwnn Abstract Syntax}

\begin{gather*}
  n \in \Int \qquad b \in \Bool \qquad \str \in \String \qquad
  x \in \Variable 
  \\ 
  \cname \in \CName \qquad \mname \in \MName
\end{gather*}

\nvsp\nvsp

\begin{align*}
  p \in \Program &\Coloneqq \wseq{\class}
  \\%[.1in]
  \class \in \Class &\Coloneqq \kw{class } \cname_1 \kw{ extends }
  \cname_2 \;\left\{\; \kw{fields }\wseq{x : \typ} \;\cdot\; \kw{methods
  }\wseq{m} \;\right\} 
  \\%[.1in]
  \typ \in \Type &\Coloneqq \intt \alt \boolt \alt \strt \alt \nullt
  \alt \cname
  \\%[.1in]
  m \in \Method &\Coloneqq \kw{def }\mname(\wseq{x : \typ}) :
  \typ_{\mtt{ret}} \; \left\{\; \seq{\stmt} \cdot \kw{return }e \;\right\}
  \\%[.1in]
  \stmt \in \Stmt &\Coloneqq x := e \alt e_1.x := e_2 \alt x :=
  e.\mname(\seq{e}) \alt x := \kw{new }\cname(\seq{e})
  \\
  &\lalt \cond{e}{\wseq{\stmt_1}}{\wseq{\stmt_2}} \alt
  \while{e}{\seq{\stmt}}
  \\%[.1in]
  e \in \Exp &\Coloneqq \uset{n} \alt \uset{b} \alt \uset{\str} \alt \nullv
  \alt x \alt e.x \alt e_1 \binop e_2 
  \\%[.1in]
  \binop \in \BinOp &\Coloneqq + \alt - \alt \times \alt \div \alt < \alt
  \leq \alt \land \alt \lor \alt = \alt \neq
\end{align*}

\paragraph{Notation.} By abuse of notation we use the vector notation
\wseq{\something} to indicate an ordered sequence of unspecified size
$n$, indexed from $0 \leq i < n$. We use the overline notation
\uset{\something} to indicate an unordered set. The length of a vector
(respectively, set) is denoted by $|\wseq{\something}|$ (respectively,
$|\uset{\something}|$).

\paragraph{Syntax Summary.} A \textit{program} consists of a sequence of
classes. A \textit{class} specifies a name, a superclass, a set of
fields (each field consisting of a variable and its type), and a set
of methods. \textit{Types} represent integers, booleans, strings,
\kw{null}, or one of the user-defined classes, respectively (i.e., the
\nullt type has a single value, called \kw{null}). A \textit{method}
specifies a name, a set of parameters (each consisting of a variable
and its type), a return type, and a body. The body of a method is a
sequence of statements terminated by a \kw{return} statement. A
\textit{statement} is an assignment, an object field update, an object
method call, new object construction, a conditional, or a while
loop. An \textit{expression} is a set of integers, a set of booleans,
a set of strings, the \kw{null} value, a variable, an object field
access, or a binary operation. We use sets of integers, booleans, and
strings to allow for nondeterministic execution without needing to
specify I/O for the language.

\paragraph{Type System.} The language is statically typed, using a
nominal type system with subtyping and recursive types. The \intt,
\boolt, and \strt types are invariant. The \nullt type is a subtype of
all classes. All classes are subtypes of a builtin class called
\TopClass that has no fields or methods. Every type has a default
value of that type, which is used to initialize object fields and any
method parameters that don't receive an argument. The default value of
\intt is 0; the default value of \boolt is \false; the default value
of \strt is \ttt{""}; the default value of \nullt and all class types
is \kw{null}.

\paragraph{Assumptions.} All methods implicitly have a parameter
\self, which contains the address of the object that the method was
called on (i.e., the \ttt{this} parameter in C++ and Java). Method
calls can provide fewer arguments than there are parameters; the extra
parameters are given default values (this is how methods can declare
local variables). All classes contain a constructor method, defined as
a method with the same name as that class; this method is called
whenever an object of that class is created using \kw{new}. All
constructors should end with $\kw{return }\self$. When a program is
executed, it takes the first class in the program and calls its
constructor as the entry point to the program. In the sequence of
class definitions, a superclass must be defined before any class that
inherits from it.

\paragraph{Concrete Syntax.} The concrete syntax allows various
shortcuts by making some parts of the syntax optional. If these
optional parts are left out, the parser will fill them in with default
syntax to match the required abstract syntax. In particular:

\begin{itemize}
\item A class doesn't need to declare any fields or methods; if they
  are left out then the corresponding part of the abstract syntax will
  be the empty set.

\item A class doesn't need to specify a superclass using \kw{extend};
  if this is left out then in the abstract syntax the class will
  extend \TopClass.

\item A method doesn't need to end with a \kw{return}; if this is left
  out the abstract syntax will use $\kw{return }\self$.

\item If a method doesn't syntactically contain a \kw{return}, it
  doesn't need to specify the method's return type. The return type
  will be inferred in the abstract syntax to be the method's
  containing class.

\item A method call doesn't need to assign the result to a
  variable. If there is no assignment, the abstract syntax will assign
  the return value to a dummy variable.

\item An \kw{if} statement doesn't need to have an \kw{else}
  clause. If it is left out, the abstract syntax will contain an
  \kw{else} clause with a single statement that effectively is a
  no-op.
\end{itemize}

\pagebreak

\section{Lwnn Concrete Semantics}

We describe the semantic domains that constitute a state of the
transition system (Section~\ref{ssec:cdomains}), state transition
rules (Section~\ref{ssec:crules}), and the helper functions used by
the transition rules (Section~\ref{ssec:chelpers}).

\subsection{Concrete Semantic Domains}
\label{ssec:cdomains}

\begin{align*}
  \state \in \State &= \Classes \times \sopt{\Stmt} \times \Local \times
  \Heap \times \sseq{\Kont}
  \\
  \classes \in \Classes &= \CName \to ((\Variable \to \Value) \times
  (\MName \to \Method))
  \\
  \local \in \Local &= \Variable \to \Value
  \\
  \heap \in \Heap &= \Address \to \Object
  \\
  r \in \Ref &= \Address \,\uplus\, \left\{\nullv\right\}
  \\
  v \in \Value &= \Int \,\uplus\, \Bool \,\uplus\, \String \,\uplus\,
  \Ref
  \\
  a \in \Address &= \Nat
  \\
  o \in \Object &= \CName \times (\Variable \to \Value)
  \\
  \kont \in \Kont &= \sstmtk{\Stmt} \;\uplus\; \swhilek{\Exp}{\sseq{\Stmt}}
  \;\uplus\; \sretk{\Variable}{\Exp}{\Local}
\end{align*}

\paragraph{Notation.} We borrow notation from formal languages:
$\something^?$ means 0 or 1 instances; $\something^\star$ means an ordered
sequence of 0 or more instances; $\something^+$ means an ordered sequence
of 1 or more instances. The $\uplus$ operator means disjoint union.

\paragraph{Domains Summary.} A state consists of the class definitions
(which are invariant across all states), an optional statement to be
processed, a map from the current method's local variables to their
values, a heap mapping addresses to objects, and a continuation
stack. The class definitions map each class name to a pair of maps;
the first maps the class's fields to their default values, and the
second maps the class's method names to the method
definitions. Language values are integers, booleans, strings, object
references. An object reference is either an address or \kw{null}. An
object is a tuple of the object's class name and a map from the class
field's to their values for this object. The continuation stack is a
sequence of \kw{stmtK} continuations (holding statements to be
processed), \kw{whileK} continuations (holding the guard and body of a
currently executing while loop, so we can start the next iteration),
and \kw{retK} continuations (holding the variable to receive the
callee method's return value, the expression whose value should be
returned from the callee, and the caller method's local variables so
that we can restore them when the callee returns).

\subsection{Concrete Transition Rules}
\label{ssec:crules}

\begin{longtable}{R|L|L|C|C|C|L}
  \caption{The concrete transition relation. Each rule describes how
    to take one concrete state $(\classes, \sopt{\stmt}, \local,
    \heap, \kont \cdot \wseq{\kont_1})$ to the next concrete state
    $(\classes, \sopt{\stmt}_{\mtt{new}}, \local_{\mtt{new}},
    \heap_{\mtt{new}}, \wseq{\kont_\mtt{new}})$. The
    $\sopt{s}$ notation means a statement may or may not exist; we use
    $\nada$ to indicate that there is no statement. The $\cdot$
    operator used for the continuation stack indicates appending
    sequences; thus $\kont$ is the top of the continuation stack in
    the source state and $\wseq{\kont_1}$ is the rest of that
    continuation stack.}
  \\ \hline
  \mtt{no.} & \sopt{\stmt} & \textbf{premises} & \sopt{\stmt}_{\mtt{new}} &
  \local_{\mtt{new}} & \heap_{\mtt{new}} & \wseq{\kont_\mtt{new}}
  \\ \hline
  1 & x := e & \evalTo{e} = v & \nada & \local[x \mapsto v] & \heap &
  \wseq{\kont_1}
  \\
  2 & e_1.x := e_2 & \evalTo{e_1} = a,\, \evalTo{e_2} = v,\; o =
  \heap(a)[x \mapsto v] & \nada & \local & \heap[a \mapsto o] &
  \wseq{\kont_1} 
  \\
  3 & x := e.\mname(\seq{e}) & \left(\local_1,
  \wseq{\kont_2}\right) = \funcall\left(\classes, x, \evalTo{e},
  \heap, \mname, \wseq{\evalTo{e}}, \local\right) & \nada & \local_1 &
  \heap & \wseq{\kont_2} \cdot \wseq{\kont_1} 
  \\
  4 & x := \kw{new }\cname(\seq{e}) & \left(\local_1, \heap_1,
  \wseq{\kont_2}\right) = \construct\left(\classes, x, \cname,
  \wseq{\evalTo{e}}, \local, \heap\right) & \nada & \local_1 & \heap_1 &
  \wseq{\kont_2} \cdot \wseq{\kont_1} 
  \\
  5 & \cond{e}{\wseq{\stmt_1}}{\wseq{\stmt_2}} & \evalTo{e} = \true &
  \nada & \local & \heap & \tostmtk(\wseq{\stmt_1}) \cdot \wseq{\kont_1} 
  \\
  6 & \cond{e}{\wseq{\stmt_1}}{\wseq{\stmt_2}} & \evalTo{e} = \false &
  \nada & \local & \heap & \tostmtk(\wseq{\stmt_2}) \cdot \wseq{\kont_1} 
  \\
  7 & \while{e}{\seq{\stmt}} & \evalTo{e} = \true & \nada & \local &
  \heap & \tostmtk(\seq{\stmt}) \cdot \kont \cdot \wseq{\kont_1} 
  \\
  8 & \while{e}{\seq{\stmt}} & \evalTo{e} = \false & \nada & \local &
  \heap & \wseq{\kont_1} 
  \\
  9 & \nada & \kont = \retk{x}{e}{\local_1},\, \evalTo{e} = v &
  \nada & \local_1[x \mapsto v] & \heap & \wseq{\kont_1} 
  \\
  10 & \nada & \kont = \stmtk{\stmt_1} & \stmt_1 & \local & \heap &
  \wseq{\kont_1}  
  \\
  11 & \nada & \kont = \whilek{e}{\seq{\stmt}},\, \evalTo{e} = \true
  & \nada & \local & \heap & \tostmtk(\seq{\stmt}) \cdot \kont \cdot
  \wseq{\kont_1}  
  \\
  12 & \nada & \kont = \whilek{e}{\seq{\stmt}},\, \evalTo{e} = \false
  & \nada & \local & \heap & \wseq{\kont_1} 
\end{longtable}

\paragraph{Notation.} We use \evalTo{e} as shorthand for $\eeval(e,
\local, \heap)$ when \local and \heap are obvious from context. We use
$\sopt{s}$ to indicate 0 or 1 statements; \nada means there is no
statement. For any map $X$, the notation $X[a \mapsto b]$ means a new
map that is exactly the same as $X$ except that $a$ maps to $b$. We
abuse notation for objects by using this map update notation to update
object fields in rule 2, even though technically objects are a pair of
class name and a map. We use \proj{i}{\mtt{tuple}} to project out the
$i$th element of \mtt{tuple}.

\subsection{Concrete Helper Functions}
\label{ssec:chelpers}

We describe the helper functions used by the transition rules. The
functions are listed in alphabetical order. Note that in several
places we implictly assume that a reference value must be an address
rather then \nullv; this means that the behavior if the reference is
actually \nullv is undefined.

\paragraph{Notation.}  The notation $\mtt{map_1}[\mtt{map_2}]$ is
shorthand for updating $\mtt{map_1}$ with each entry in $\mtt{map_2}$
in turn. Recall that we use \proj{i}{\mtt{tuple}} to project out the
$i$th element of \mtt{tuple}.

\subsubsection{\fbox{$\eeval(e, \local, \heap)$\ \  a.k.a.\ \ \evalTo{e}}}

This function describes how to evaluate expressions to values. Note
that sets of integers/booleans/strings are evaluated by
nondeterministically selecting an element from that set. Variables are
looked up in the locals map; object field access gets the address of
an object and then looks up the given field's value in that object;
binary operators recursively evaluate the operands and then apply the
appropriate operation to the result (the operators are described below).

\nvsp
\begin{flalign*}
  \eeval : \Exp \times \Local \times \Heap \to \Value&&
\end{flalign*}

\nvsp\nvsp\nvsp\nvsp

\begin{flalign*}
  \eeval(e, \local, \heap) &=&
  \\
  &\quad
  \begin{cases}
    n &\text{if }e = \uset{n},\, n \in \uset{n}
    \\
    b &\text{if }e = \uset{b},\, b \in \uset{b}
    \\
    \str &\text{if }e = \uset{\str},\, \str \in \uset{\str}
    \\
    \nullv &\text{if }e = \nullv
    \\
    \local(x) &\text{if }e = x
    \\
    \mtt{fields}(x) &\text{if }e = e_1.x,\, \evalTo{e_1} = a,\,
    \proj{2}{\heap(a)} = \mtt{fields}
    \\
    \evalTo{e_1} \binop \evalTo{e_2} &\text{if }e = e_1 \binop e_2
  \end{cases}
\end{flalign*}

\paragraph{Operators on Integers.}
$\{+, -, \times, \div\} : \Int \times \Int \to \Int$ are the standard
(unbounded width) integer arithmetic operators.

\paragraph{Operators on Booleans.}
$\{\land, \lor\} : \Bool \times \Bool \to \Bool$ are the standard
logical \textsc{and} and \textsc{or} operators.

\paragraph{Operators and Strings.}
$+ : \String \times \String \to \String$ is string
concatenation. $\{<, \leq\} : \String \times \String \to \Bool$ are
strict and reflexive lexicographic string comparison, respectively.

\paragraph{Operators on Values.}
$\{=, \neq\} : \Value \times \Value \to \Bool$ are equality and
inequality of values, respectively.

\subsubsection{\fbox{\funcall}}

This function describes how to process a method call. Note that the
current locals map is saved in the \kw{retK} continuation to be
restored once the callee returns, and that \self is mapped to the
object's address in the new locals map. Any arguments are copied to
their respective parameters; any parameters not given an argument are
mapped to their type's default value.

\nvsp
\begin{flalign*}
  &\funcall \in \Classes \times \Variable \times \Address \times
  \Heap \times \MName \times \sseq{\Value} \times \Local \to \Local
  \times \sseq{\Kont}&
  \\
  &\funcall\left(\classes, x, a, \heap, \mname, \seq{v}, \local\right) =
  \left(\local_1, \wseq{\kont_1}\right) \qquad\text{where}&
  \\
  &\qquad\cname = \proj{1}{\heap(a)}
  \\
  &\qquad\mtt{methods} = \proj{2}{\classes(\cname)}
  \\
  &\qquad\mtt{methods}(\mname) = \kw{def }\mname(\wseq{x : \typ}) :
  \typ_{\mtt{ret}} \; \{\; \seq{\stmt} \cdot \kw{return }e \;\}
  \\
  &\qquad\wseq{\kont_1} = \tostmtk(\seq{\stmt}) \cdot
  \retk{x}{e}{\local} 
  \\
  &\qquad \local_1 = [\self \mapsto a] \cup [\;x_i \mapsto v \alt 0
    \leq i < |\seq{v}| \implies v = v_i,\; |\seq{v}| \leq i < |\wseq{x
      : \typ}| \implies v = \defval(\typ_i)\;] 
\end{flalign*}

\subsubsection{\fbox{\construct}}

This function describes how to create a new object. It retrieves the
class's fields and their default values from the class definitions to
create a new object, then allocates a fresh address and creates a new
heap that maps the address to the new object. It then retrieves the
constructor method for that class and proceeds as if for a method
call. Note that since constructors must end in $\kw{return }\self$,
$x$ will get the new object's address when the constructor returns.

\nvsp
\begin{flalign*}
  &\construct \in \Classes \times \Variable \times \CName \times
  \sseq{\Value} \times \Local \times \Heap \to \Local \times \Heap
  \times \sseq{\Kont}&
  \\
  &\construct\left(\classes, x, \mname, \seq{v}, \local, \heap\right) = 
  \left(\local_1, \heap_1, \wseq{\kont_1}\right) \qquad\text{where}&
  \\
  &\qquad a \text{ is a fresh address}
  \\
  &\qquad o = (\cname, \proj{1}{\classes(\cname)})
  \\
  &\qquad \heap_1 = \heap[a \mapsto o]
  \\
  &\qquad\mtt{methods} = \proj{2}{\classes(\cname)}
  \\
  &\qquad\mtt{methods}(\cname) = \kw{def }\cname(\wseq{x : \typ}) :
  \typ_{\mtt{ret}} \; \{\; \seq{\stmt} \cdot \kw{return }\self \;\}
  \\
  &\qquad\wseq{\kont_1} = \tostmtk(\seq{\stmt}) \cdot
  \retk{x}{\self}{\local} 
  \\
  &\qquad \local_1 = [\self \mapsto a] \cup [\;x_i \mapsto v \alt 0
    \leq i < |\seq{v}| \implies v = v_i,\; |\seq{v}| \leq i < |\wseq{x
    : \typ}| \implies v = \defval(\typ_i)\;] 
\end{flalign*}

\subsubsection{\fbox{\defval}}

This function maps each type to that type's default value.

\nvsp
\begin{flalign*}
  &\defval \in \Type \to \Value
  \\
  &\defval(\typ) =&
  \\
  &\quad
  \begin{cases}
    0 &\text{if }\typ = \intt
    \\
    \false &\text{if }\typ = \boolt
    \\
    \ttt{""} &\text{if }\typ = \strt
    \\
    \nullv &\text{otherwise}
  \end{cases}
\end{flalign*}

\subsubsection{\fbox{\initstate}}

This function takes the program and generates the initial state. It
creates the class definitions and calls the constructor of the first
class in the program as the starting point of the program's
execution. It uses a secondary helper function \initclass to convert a
syntactic class definition into a semantic class definition. \fold is
the standard functional fold-left function that takes a function, an
initial value, and a sequence and applies the function to each element
of the sequence, passing the result of each function call to the next
call in the chain.

\nvsp
\begin{flalign*}
  &\initstate \in \Program \to \State
  \\
  &\initstate(p) = (\classes, \nada, \local, \heap, \seq{\kont})
  \qquad\text{where}&
  \\
  &\qquad \classes = \fold(\; (\mtt{acc}, \class \Rightarrow
  \mtt{acc} \cup [\class.\cname_1 \mapsto \initclass(\class)]),\;
  [\TopClass \mapsto (\emptyset, \emptyset)],\; p)
  \\
  &\qquad \cname \text{ is the name of the first class in }p
  \\
  &\qquad a \text{ is a fresh address}
  \\
  &\qquad o = (\cname, \proj{1}{\classes(\cname)})
  \\
  &\qquad \heap = [a \mapsto o]
  \\
  &\qquad\mtt{methods} = \proj{2}{\classes(\cname)}
  \\
  &\qquad\mtt{methods}(\cname) = \kw{def }\cname(\wseq{x : \typ}) :
  \typ_{\mtt{ret}} \; \{\; \seq{\stmt} \cdot \kw{return }\self \;\}
  \\
  &\qquad\seq{\kont} = \tostmtk(\seq{\stmt})
  \\
  &\qquad \local = [\self \mapsto a] \cup [\;x_i \mapsto
    \defval(\typ_i) \alt 0 \leq i < |\wseq{x : \typ}| \;] 
\end{flalign*}

\nvsp\nvsp
\begin{flalign*}
  &\initclass \in \Classes \times \Class \to (\Variable \to \Value)
  \times (\MName \to \Method)
  \\
  &\initclass(\classes, \class) = (\mtt{fields},\, \mtt{methods})
  \qquad\text{where}&
  \\
  &\qquad \class = \kw{class } \cname_1 \kw{ extends } \cname_2 \;\{\;
  \kw{fields }\wseq{x : \typ} \;\cdot\; \kw{methods }\wseq{m} \;\}
  \\
  &\qquad \mtt{superflds} = \proj{1}{\classes(\cname_2)}
  \\
  &\qquad \mtt{supermethods} = \proj{2}{\classes(\cname_2)}
  \\
  &\qquad \mtt{localflds} = [\; x_i \mapsto \defval(\typ_i) \alt 0
    \leq i < |\wseq{x : \typ}| \;]
  \\
  &\qquad \mtt{localmethods} = [\; m_j.\mname \mapsto m_j \alt 0 \leq
    j < |\wseq{m}| \;]
  \\
  &\qquad \mtt{fields} = \mtt{superflds}[\mtt{localflds}]
  \\
  &\qquad \mtt{methods} = \mtt{supermethods}[\mtt{localmethods}]
\end{flalign*}

\subsubsection{\fbox{\tostmtk}}

This function maps a sequence of statements to a sequence of
\kw{stmtK} continuations containing those statements.

\nvsp
\begin{flalign*}
  &\tostmtk \in \sseq{\Stmt} \to \sseq{\Kont}
  \\
  &\tostmtk(\seq{\stmt}) = \seq{\kont}
  \qquad\text{where } \kont_i = \stmtk{\stmt_i} \text{ for }0 \leq i <
  |\seq{\stmt}|&
\end{flalign*}

\pagebreak

\section{Lwnn Abstract Semantics}

We describe the semantic domains that constitute an abstract state
(Section~\ref{ssec:adomains}), abstract state transition rules
(Section~\ref{ssec:arules}), and the helper functions used by the
abstract transition rules (Section~\ref{ssec:ahelpers}).

\subsection{Abstract Semantic Domains}
\label{ssec:adomains}

\begin{gather*}
  \an \in \AInt \qquad \ab \in \ABool \qquad \astr \in \AString \qquad
  \aad \in \AAddress \qquad \abinop \in \ABinOp
\end{gather*}

\nvsp\nvsp
\begin{align*}
  \astate \in \AState &= \AClasses \times \sopt{\Stmt} \times \ALocal \times
  \AHeap \times \sseq{\AKont\ \!} 
  \\
  \aclasses \in \AClasses &= \CName \to \left(\left(\Variable \to
  \AValue\right) \times \left(\MName \to \Method\right)\right)
  \\
  \alocal \in \ALocal &= \left(\Variable \to \AValue\right) \times
  \power\left(\sseq{\AKont\ \!}\right)
  \\
  \aheap \in \AHeap &= \AAddress \to \AObject
  \\
  \aref \in \ARef &= \power\left(\AAddress \cup \{ \nullv \}\right) 
  \\
  \av \in \AValue &= \AInt \,\uplus\, \ABool \,\uplus\, \AString \,\uplus\,
  \ARef
  \\
  \ao \in \AObject &= \CName \times \left(\Variable \to \AValue\right)
  \\
  \akont \in \AKont &= \asstmtk{\Stmt} \;\uplus\; \aswhilek{\Exp}{\sseq{\Stmt}}
  \;\uplus\; \asretk{\Variable}{\Exp}{\ALocal} \;\uplus\; \asfink
\end{align*}

\paragraph{Domains Summary.} We leave the abstract number, boolean,
string, and address domains unspecified. The abstract boolean
operators depend on the specific abstractions chosen for these
domains. We augment the \ALocal domain to include a set of
continuation stacks and add a semantic continuation $\asfink$; these
are used to provide a level of indirection for handling method calls
that is necessary for computability. An abstract reference is a set
containing abstract addresses and/or \nullv---we do this because we
are over-approximating the concrete semantics, in which an object
reference would be only a single address or \nullv.

\subsection{Abstract Transition Rules}
\label{ssec:arules}

\begin{longtable}{R|L|L|C|C|C|L}
  \caption{The abstract transition relation. Each rule describes how
    to take one abstract state $(\aclasses, \sopt{\stmt}, \alocal,
    \aheap, \akont \cdot \wseq{\akont_1})$ to the next abstract state
    $(\aclasses, \sopt{\stmt}_{\mtt{new}}, \alocal_{\mtt{new}},
    \aheap_{\mtt{new}}, \wseq{\akont_\mtt{new}})$. The
    $\sopt{s}$ notation means a statement may or may not exist; we use
    $\nada$ to indicate that there is no statement. The $\cdot$ operator
  used for the continuation stacks indicates appending sequences; thus
  $\akont$ is the top of the continuation stack in the source state and
  $\wseq{\akont_1}$ is the rest of that continuation stack.}
  \\ \hline
  \mtt{no.} & \sopt{\stmt} & \textbf{premises} & \sopt{\stmt}_{\mtt{new}} &
  \alocal_{\mtt{new}} & \aheap_{\mtt{new}} & \wseq{\akont_\mtt{new}}
  \\ \hline
  1 & x := e & \aevalTo{e} = \av & \nada & \alocal[x \mapsto \av] &
  \aheap & \wseq{\akont_1} 
  \\
  2 & e_1.x := e_2 & \aheap_1 = \aupdate(\aheap, \aevalTo{e_1}, x,
  \aevalTo{e_2}) & \nada & \alocal & \aheap_1 & \wseq{\akont_1} 
  \\
  3 & x := e.\mname(\seq{e}) & \left(\alocal_1, \wseq{\akont_2}\right) \in
  \afuncall\left(\aclasses, x, \aevalTo{e}, \aheap,  \mname,
  \wseq{\aevalTo{e}}, \alocal, \wseq{\akont_1} \right) & \nada &
  \alocal_1 & \aheap & \wseq{\akont_2} 
  \\
  4 & x := \kw{new }\cname(\seq{e}) & \left(\alocal_1, \aheap_1,
  \wseq{\akont_2}\right) = \aconstruct\left(\aclasses, x, \cname,
  \wseq{\aevalTo{e}}, \alocal, \aheap, \wseq{\akont_1} \right) & \nada &
  \alocal_1 & \aheap_1 & \wseq{\akont_2} 
  \\
  5 & \cond{e}{\wseq{\stmt_1}}{\wseq{\stmt_2}} & \true \in
  \gamma_b(\aevalTo{e}) & \nada & \alocal & \aheap &
  \atostmtk(\wseq{\stmt_1}) \cdot \wseq{\akont_1} 
  \\
  6 & \cond{e}{\wseq{\stmt_1}}{\wseq{\stmt_2}} & \false \in
  \gamma_b(\aevalTo{e}) & \nada & \alocal & \aheap &
  \atostmtk(\wseq{\stmt_2}) \cdot \wseq{\akont_1}  
  \\
  7 & \while{e}{\seq{\stmt}} & \true \in \gamma_b(\aevalTo{e}) &
  \nada & \alocal & \aheap & \atostmtk(\seq{\stmt}) \cdot \akont
  \cdot \wseq{\akont_1} 
  \\
  8 & \while{e}{\seq{\stmt}} & \false \in \gamma_b(\aevalTo{e}) &
  \nada & \alocal & \aheap & \wseq{\akont_1} 
  \\
  9 & \nada & \akont = \afink,\, \aretk{x}{e}{\alocal_1} \cdot
  \wseq{\akont_2} \in \proj{2}{\alocal},\, \aevalTo{e} = \av
  & \nada & \alocal_1[x \mapsto \av] & \aheap & \wseq{\akont_2}
  \\
  10 & \nada & \akont = \astmtk{\stmt_1} & \stmt_1 & \alocal & \aheap &
  \wseq{\akont_1}  
  \\
  11 & \nada & \akont = \awhilek{e}{\seq{\stmt}},\, \true \in
  \gamma_b(\aevalTo{e}) & \nada & \alocal & \aheap &
  \atostmtk(\seq{\stmt}) \cdot \akont \cdot \wseq{\akont_1} 
  \\
  12 & \nada & \akont = \awhilek{e}{\seq{\stmt}},\, \false \in
  \gamma_b(\aevalTo{e}) & \nada & \alocal & \aheap & \wseq{\akont_1}
\end{longtable}

\paragraph{Transitions Summary.} Rules 5--8 and 11--12 use
$\gamma_b$, the boolean concretization operator that maps an abstract
boolean value to the corresponding set of concrete boolean values. In
rules 2--3 the helper functions use the indirection provided by
$\ALocal$ to store the current continuation stack inside the
continuation stack set of $\alocal_1$ for a method/constructor call,
and rule 9 restores the continuation stack from that set when
returning from the callee.

\subsection{Abstract Helper Functions}
\label{ssec:ahelpers}

We describe the helper functions used by the abstract transition
rules. The functions are listed in alphabetical order. Note that in
several places we implicitly ignore the possibility that an abstract
reference value may contain \nullv; since these operations on \nullv
are undefined in the concrete semantics, it is sound to ignore them in
the abstract semantics.

\subsubsection{\fbox{$\aeeval(e, \alocal, \aheap)$\ \  a.k.a.\ \ \aevalTo{e}}}

This function describes how to evaluate expressions to abstract
values. It uses the number abstaction function $\alpha_n$, the boolean
abstraction function $\alpha_b$, the string abstraction function
$\alpha_\str$, and the reference abstraction function
$\alpha_r$. Field access uses the helper function \alookup. The
abstract binary operators are left unspecified because they depend on
the abstractions chosen for the abstract value domains.

\nvsp
\begin{flalign*}
  \aeeval : \Exp \times \ALocal \times \AHeap \to \AValue&&
\end{flalign*}

\nvsp\nvsp\nvsp
\begin{flalign*}
  \aeeval(e, \alocal, \aheap) &=&
  \\
  &\quad
  \begin{cases}
    \an &\text{if }e = \uset{n},\, \alpha_n(\uset{n}) = \an
    \\
    \ab &\text{if }e = \uset{b},\, \alpha_b(\uset{b}) = \ab
    \\
    \astr &\text{if }e = \uset{\str},\, \alpha_\str(\uset{\str}) = \astr
    \\
    \alpha_r(\nullv) &\text{if }e = \nullv
    \\
    \alocal(x) &\text{if }e = x
    \\
    \alookup(\aevalTo{e_1}, x, \aheap) &\text{if }e = e_1.x
    \\
    \aevalTo{e_1} \;\abinop\; \aevalTo{e_2} &\text{if }e = e_1 \binop e_2
  \end{cases}
\end{flalign*}

\subsubsection{\fbox{\afuncall}}

This function describes how to abstractly process a method call. It
returns a set of $(\ALocal, \sseq{\AKont\ \!})$ pairs because (due to
inheritance and subtype polymorphism) there could be more than one
possible method being called. The returned $\alocal_1$'s also contain
the current continuation stack, which will be restored once the callee
returns; this extra level of indirection is necessary for
computability.

\nvsp
\begin{flalign*}
  &\afuncall \in \AClasses \times \Variable \times \power(\AAddress)
  \times \AHeap \times \MName \times \sseq{\AValue\ \!} \times \ALocal
  \times \sseq{\AKont\ \!} \to \power(\ALocal \times \sseq{\AKont\ \!})&
  \\
  &\afuncall\left(\aclasses, x, \uset{\aad}, \aheap, \mname,
  \seq{\av}, \alocal, \seq{\akont}\right) = \uset{\left(\alocal_1,
    \wseq{\akont_1}\right)} \qquad\text{where}&
  \\
  &\qquad \uset{(\aad, \cname)} = \left\{\; (\aad,\,
  \proj{1}{\heap(\aad)}) \alt \aad \in \uset{\aad} \;\right\}
  \\
  &\qquad \uset{(\aad, m)} = \left\{\; (\aad,\,
  \proj{2}{\aclasses(\cname)}(\mname)) \alt (\aad, \cname) \in
  \uset{(\aad, \cname)} \;\right\}
  \\
  &\qquad \uset{\left(\alocal_1, \wseq{\akont_1}\right)} = \left\{\;
  \left(\alocal_{\aad}, \wseq{\akont_{\aad}}\right) \alt \left(\aad,\,
  \kw{def }\mname(\wseq{x : \typ}) : \typ_{\mtt{ret}} \; \{\;
  \seq{\stmt} \cdot \kw{return }e \;\}\right) \in \uset{(\aad, m)}
  \;\right\} \qquad\text{where}
  \\
  &\qquad\qquad \alocal_{\aad} = \left([\self \mapsto \aad] \cup
    [\;x_i \mapsto \av \alt 0 \leq i < |\seq{\av}| \implies \av =
      \av_i,\; |\seq{\av}| \leq i < |\wseq{x : \typ}| \implies \av =
      \adefval(\typ_i)\;],\; \left\{\,\aretk{x}{\self}{\alocal}
      \cdot \seq{\akont}\,\right\}\right)
  \\
  &\qquad\qquad \wseq{\akont_{\aad}} = \atostmtk(\seq{\stmt}) \cdot \afink
\end{flalign*}

\subsubsection{\fbox{\aconstruct}}

This function describes how to create a new abstract object. It leaves
determination of the abstract address at which to allocate the object
unspecified; this will depend on the heap model used by the
analysis. That abstract address may be \textit{strong} (i.e.,
correspond to a single concrete address) or \textit{weak} (otherwise);
in the first case $\aheap_1$ is updated with the new object, in the
second case $\aheap_1$ is updated with the join of the new object and
any object currently allocated at that address---since the set of
abstract addresses must be finite, the analysis may have to reuse the
same abstract address for different objects. The $\gamma_a$ operator
is the abstract address concretization operator; note that the
\textit{implementation} of this helper function \textit{should not}
actually use the concretization operator (which would be
uncomputable).

\nvsp
\begin{flalign*}
  &\aconstruct \in \AClasses \times \Variable \times \CName \times
  \sseq{\AValue\ \!} \times \ALocal \times \AHeap \times
  \sseq{\AKont\ \!} \to \ALocal \times \AHeap \times \sseq{\AKont\ \!}&
  \\
  &\aconstruct\left(\aclasses, x, \mname, \seq{\av}, \alocal, \aheap,
  \seq{\akont}\right) = \left(\alocal_1, \aheap_1,
  \wseq{\akont_1}\right) \qquad\text{where}& 
  \\
  &\qquad \aad \text{ depends on the heap model}
  \\
  &\qquad \ao = (\cname, \proj{1}{\aclasses(\cname)})
  \\
  &\qquad \aheap_1 = 
  \begin{cases}
    \aheap[\aad \mapsto \ao] & \text{if }|\gamma_a(\aad)| = 1
    \\
    \aheap[\aad \mapsto \ao \sqcup \aheap(\aad)] & \text{otherwise}
  \end{cases}
  \\
  &\qquad\mtt{methods} = \proj{2}{\aclasses(\cname)}
  \\
  &\qquad\mtt{methods}(\cname) = \kw{def }\cname(\wseq{x : \typ}) :
  \typ_{\mtt{ret}} \; \{\; \seq{\stmt} \cdot \kw{return }\self \;\}
  \\
  &\qquad\wseq{\akont_1} = \atostmtk(\seq{\stmt}) \cdot
  \afink
  \\
  &\qquad \alocal_1 = \left([\self \mapsto \aad] \cup [\;x_i \mapsto
    \av \alt 0 \leq i < |\seq{\av}| \implies \av = \av_i,\;
    |\seq{\av}| \leq i < |\wseq{x : \typ}| \implies \av =
    \adefval(\typ_i)\;],\; \left\{\,\aretk{x}{\self}{\alocal} \cdot
  \seq{\akont}\,\right\}\right)
\end{flalign*}

\subsubsection{\fbox{\adefval}}

This function maps each type to that type's default abstract value. It
uses the number abstaction function $\alpha_n$, the boolean
abstraction function $\alpha_b$, the string abstraction function
$\alpha_\str$, and the reference abstraction function $\alpha_r$.

\nvsp
\begin{flalign*}
  &\adefval \in \Type \to \AValue
  \\
  &\adefval(\typ) =&
  \\
  &\quad
  \begin{cases}
    \alpha_n(0) &\text{if }\typ = \intt
    \\
    \alpha_b(\false) &\text{if }\typ = \boolt
    \\
    \alpha_\str(\ttt{""}) &\text{if }\typ = \strt
    \\
    \alpha_r(\nullv) &\text{otherwise}
  \end{cases}
\end{flalign*}

\subsubsection{\fbox{\ainitstate}}

This function takes the program and generates the initial abstract
state. It is similar to the concrete version except that it uses the
corresponding abstractions of the concrete values.

\nvsp
\begin{flalign*}
  &\ainitstate \in \Program \to \AState
  \\
  &\ainitstate(p) = \left(\aclasses, \nada, \alocal, \aheap,
  \seq{\akont}\right) \qquad\text{where}&
  \\
  &\qquad \aclasses = \fold(\; (\mtt{acc}, \class \Rightarrow
  \mtt{acc} \cup [\class.\cname_1 \mapsto \ainitclass(\class)]),\;
  [\TopClass \mapsto (\emptyset, \emptyset)],\; p)
  \\
  &\qquad \cname \text{ is the name of the first class in }p
  \\
  &\qquad \aad \text{ is a fresh abstract address}
  \\
  &\qquad \ao = (\cname, \proj{1}{\aclasses(\cname)})
  \\
  &\qquad \aheap = [\aad \mapsto \ao]
  \\
  &\qquad\mtt{methods} = \proj{2}{\aclasses(\cname)}
  \\
  &\qquad\mtt{methods}(\cname) = \kw{def }\cname(\wseq{x : \typ}) :
  \typ_{\mtt{ret}} \; \{\; \seq{\stmt} \cdot \kw{return }\self \;\}
  \\
  &\qquad\seq{\akont} = \atostmtk(\seq{\stmt})
  \\
  &\qquad \alocal = \left([\self \mapsto \aad] \cup [\;x_i \mapsto
    \adefval(\typ_i) \alt 0 \leq i < |\wseq{x : \typ}| \;],\; \emptyset\right)
\end{flalign*}

\nvsp\nvsp
\begin{flalign*}
  &\ainitclass \in \AClasses \times \Class \to (\Variable \to \AValue)
  \times (\MName \to \Method)
  \\
  &\ainitclass(\aclasses, \class) = (\mtt{fields},\, \mtt{methods})
  \qquad\text{where}&
  \\
  &\qquad \class = \kw{class } \cname_1 \kw{ extends } \cname_2 \;\{\;
  \kw{fields }\wseq{x : \typ} \;\cdot\; \kw{methods }\wseq{m} \;\}
  \\
  &\qquad \mtt{superflds} = \proj{1}{\classes(\cname_2)}
  \\
  &\qquad \mtt{supermethods} = \proj{2}{\classes(\cname_2)}
  \\
  &\qquad \mtt{localflds} = [\; x_i \mapsto \adefval(\typ_i) \alt 0
    \leq i < |\wseq{x : \typ}| \;]
  \\
  &\qquad \mtt{localmethods} = [\; m_i.\mname \mapsto m_i \alt 0 \leq
    i < |\wseq{m}| \;]
  \\
  &\qquad \mtt{fields} = \mtt{superflds}[\mtt{localflds}]
  \\
  &\qquad \mtt{methods} = \mtt{supermethods}[\mtt{localmethods}]
\end{flalign*}

\subsubsection{\fbox{\alookup}}

This function takes a set of object addresses, an object field, and a
heap, and returns the join of all the abstract values of that field in
the objects at those addresses.

\nvsp
\begin{flalign*}
  &\alookup \in \power(\AAddress) \times \Variable \times \AHeap \to
  \AValue
  \\
  &\alookup\left(\uset{\aad}, x, \aheap\right) = \bigsqcup \left\{\;
  \av \alt \aad \in \uset{\aad},\; \proj{2}{\aheap(\aad)}(x) = \av
  \;\right\}& 
\end{flalign*}

\subsubsection{\fbox{\atostmtk}}

This function maps a sequence of statements to a sequence of
$\widehat{\kw{stmtK}}$ continuations containing those statements.

\nvsp
\begin{flalign*}
  &\atostmtk \in \sseq{\Stmt} \to \sseq{\AKont\ \!}
  \\
  &\atostmtk(\seq{\stmt}) = \seq{\akont}
  \qquad\text{where } \akont_i = \astmtk{\stmt_i} \text{ for }0 \leq i
  < |\seq{\stmt}|&  
\end{flalign*}

\subsubsection{\fbox{\aupdate}}

This function takes a set of object addresses, an object field, the
new abstract value for that field, and a heap and returns an updated
heap that has suitably updated the objects at those addresses. It
performs a \textit{strong} or \textit{weak} update of those objects
depending on if the given abstract addresses map to a single concrete
address or not. The $\gamma_a$ operator is the abstract address
concretization operator; note that the \textit{implementation} of this
helper function \textit{should not} actually use the concretization
operator (which would be uncomputable).

\nvsp
\begin{flalign*}
  &\aupdate \in \AHeap \times \power(\AAddress) \times \Variable
  \times \AValue \to \AHeap
  \\
  &\aupdate\left(\aheap, \uset{\aad}, x, \av\right) =&
  \\
  &\qquad 
  \begin{cases}
    \aheap[\aad \mapsto \ao[x \mapsto \av]]
    &\text{if }\uset{\aad} = \{\aad\},\, |\gamma_a(\aad)| = 1,\,
    \aheap(\aad) = \ao
    \\
    \aheap[\aad \mapsto \ao[x \mapsto \av \sqcup \proj{2}{\ao}(x)]] &
    \text{otherwise, for }\aad \in \uset{\aad},\, \ao = \aheap(\aad)
  \end{cases}
\end{flalign*}

\end{document}
